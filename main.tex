% !TeX root = main.tex

\documentclass[a4paper, oneside]{article}
% \usepackage[margin=2cm,bottom=4cm]{geometry}
\addtolength{\oddsidemargin}{-.25in}
\addtolength{\evensidemargin}{-.25in}
\addtolength{\textwidth}{1.in}

\usepackage{tsetsko-style}
\addbibresource{ref.bib}

\title{Final Project}
\date{\today}

\author{Tsvetelin Kostadinov}
\newcommand{\univname}{Sofia University "St. Kliment Ohridski"\\Faculty of mathematics and informatics}

% FOR CODE
% Default fixed font does not support bold face
\DeclareFixedFont{\ttb}{T1}{txtt}{bx}{n}{12} % for bold
\DeclareFixedFont{\ttm}{T1}{txtt}{m}{n}{12}  % for normal

% Custom colors
\usepackage{color}
\definecolor{deepblue}{rgb}{0,0,0.5}
\definecolor{deepred}{rgb}{0.6,0,0}
\definecolor{deepgreen}{rgb}{0,0.5,0}

\newcommand\pythonstyle{\lstset{
language=Python,
basicstyle=\ttm,
morekeywords={self, stdev, assert},% Add keywords here
keywordstyle=\ttb\color{deepblue},
emph={MyClass,__init__},          % Custom highlighting
emphstyle=\ttb\color{deepred},    % Custom highlighting style
stringstyle=\color{deepgreen},
frame=tb,                         % Any extra options here
showstringspaces=false
}}

% Python environment
\lstnewenvironment{python}[1][]
{
\pythonstyle
\lstset{#1}
}
{}

% Python for external files
\newcommand\pythonexternal[2][]{{
\pythonstyle
\lstinputlisting[#1]{#2}}}

% Python for inline
\newcommand\pythoninline[1]{{\pythonstyle\lstinline!#1!}}

\begin{document}
\begin{titlepage}
    \begin{center}
        \vspace*{-2.3cm}
        \includegraphics[height=3cm]{resources/su_logo.png}

        \vspace*{.06\textheight}
        {\scshape\large \univname\par}\vspace{2.5cm}

        {\huge \bfseries{\thetitle}\par}\vspace{0.7cm}
        \textsc{\small in }\\[0.6cm]
        \textsc{\Large Parallel Programming}\\[0.5cm]\vspace{0.5cm}
        \textsc{\normalsize School year 2024/25}\\[0.6cm]\vspace{2.2cm}


        \begin{minipage}[t]{0.4\textwidth}
            \begin{flushleft} \large
                \emph{Supervisor:}\\[0.7cm]
                Prof. Krassen Stefanov\\[0.7cm]
                Sofia\\[0.5cm]
                {\large 31.01.2025}
            \end{flushleft}
        \end{minipage}
        \begin{minipage}[t]{0.4\textwidth}
            \begin{flushright} \large
                \emph{Student:}\\[0.7cm]
                Tsvetelin Kostadinov Tsetskov\\[0.5cm]
                Faculty ID: 9MI3400529\\[0.5cm]
            \end{flushright}
        \end{minipage}

    \end{center}
\end{titlepage}
\tableofcontents
\listoffigures
\listoftables
\newpage
\section{Task Statement}
Analyze the behavior of the stochastic $\pi$ approximation with parallelization.

The approaches to be compared are:
\begin{enumerate}
    \item Sequential calculation;
    \item Multiprocessing calculation without pooling;
    \item Multiprocessing with process pooling;
    \item Multiprocessing with thread pooling;
    \item Multiprocessing with MPI
\end{enumerate}

In order to be able to accurately comment on the differences between approaches, the number of iterations will be slightly altered with a factor of 100, bringing the total number of samples to 100 000 000.

TODO:
ThreadPool for IO bound jobs - not our case
Pool for CPU bound jobs - our case

If have time - ThreadPoolExecutor and ProcessPoolExecutor
% https://docs.python.org/3.6/library/concurrent.futures.html?highlight=concurrent%20futures#threadpoolexecutor
% https://docs.python.org/3.6/library/concurrent.futures.html?highlight=concurrent%20futures#processpoolexecutor

\section{Algorithms Used}
\subsection{Sequential}
The code can be found at \figref{fig:sequential}

\section{Results Description}
\section{Future Improvements}
\subsection{Sequential}
The sequential approach would benefit from using vectorised instructions in the fashion on SIMD.

Moreover, optimizing the linear case will benefit all other approaches. Inherently all other techniques deconstruct into some number of sequential calculations.
\section{Conclusion}

\appendix
\section{Code Fragments}
\begin{figure}[h!]
    \centering
    \pythonexternal{./resources/code/sequential.py}
    \caption{Sequential stochastic algorithm}
    \label{fig:sequential}
\end{figure}

\section{Timing Results}
% \begin{table}[h!]
%     \centering
%     \begin{tabular}{c | c}
%     \end{tabular}
%     \caption{Results from sequential calculation}
% \end{table}

% \printbibliography
\end{document}